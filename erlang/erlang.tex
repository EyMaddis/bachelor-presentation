\subsection{Einbindung für Erlang}

\begin{frame}
\begin{center}
\includegraphics[scale=0.5]{erlang/pics/erlang.png}
\end{center}
\end{frame}

\begin{frame}
  \frametitle{Generelles Einbinden einer Sprache}
  \begin{center}
    Änderung an 2 Stellen:\\$~$\\
    \includegraphics[scale=0.70]{erlang/pics/Spracheinbindung}
  \end{center}
\end{frame}

\begin{frame}
\frametitle{Piratenlogik für Erlang}
\begin{center}

    \includegraphics[scale=1.0]{erlang/pics/ErlangVerarbeitungB}

\end{center}    
%wie in Ruby 
%naiver ansatz 
%Finde $->$ und kopiere dahinter die line function

%in nächster Zeit folgt
%Ziel\\
%für besseres highlighting die line-Funktion mit in die vordefinierten Funktionen kopieren
%funktionsvariablen verfolgen lassen\\
%falsche pfeile erkennen, bzw richtige\\
%Diagramm
\end{frame}

\begin{frame}
\inputminted[linenos, frame=lines, tabsize=2, fontsize=\footnotesize  ]{erlang}{erlang/Beispiel.erl}
\begin{center}
 
\end{center}
%wie bei ruby schiffslogik wird davor kopiert
%und der code entsprechend verarbeitet, \\
%laufzeitfehlermeldungen ohne shell hässlich, %werden abgefangen und hübsch weitergeleitet
%debugging bild / zeigen
\end{frame}


\begin{frame}

\begin{center}
  \includegraphics[scale=0.2]{erlang/pics/clauseerror.png}
\end{center}
\end{frame}

\begin{frame}
\begin{center}
  \includegraphics[scale=0.2]{erlang/pics/caseerror.png}
\end{center}
\end{frame}

\begin{frame}
\begin{center}
\LARGE Fazit
\end{center}
\end{frame}
